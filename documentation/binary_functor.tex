%%%%%%%%%%%%%%%%%%%%%%%%%%%%%%%%%%%%%%%%%%%%%%%%%%%%%%%%%%%%%%%%%%%%%%%%%%%%%%%%%%%%%%%%%%%%%%%%%%%
%
% Primordial Machine's Math Indices Library
% Copyright (C) 2017-2019 Michael Heilmann
%
% This software is provided 'as-is', without any express or implied warranty.
% In no event will the authors be held liable for any damages arising from the
% use of this software.
%
% Permission is granted to anyone to use this software for any purpose,
% including commercial applications, and to alter it and redistribute it
% freely, subject to the following restrictions:
%
% 1. The origin of this software must not be misrepresented;
%    you must not claim that you wrote the original software.
%    If you use this software in a product, an acknowledgment
%    in the product documentation would be appreciated but is not required.
%
% 2. Altered source versions must be plainly marked as such,
%    and must not be misrepresented as being the original software.
%
% 3. This notice may not be removed or altered from any source distribution.
%
%%%%%%%%%%%%%%%%%%%%%%%%%%%%%%%%%%%%%%%%%%%%%%%%%%%%%%%%%%%%%%%%%%%%%%%%%%%%%%%%%%%%%%%%%%%%%%%%%%%

\section{\textit{concept $n,m$-ary-Functor}}
A \textit{$n,m$-ary-Functor} is a \textit{Functor} which provides an constant \texttt{operator()} with $n+m, m, n \geq 0$ parameters.\newline

\noindent{}The first $n$ parameters are inputs to the computation.
The remaining $m$ parameters are parameters to the computation.
The types of all parameters are determined by template parameters.

%\begin{example}
\noindent{}The following example provides a partial specialization of an \texttt{equal\_to\_functor} which is a $2,1$ functor.
The functor determines wether two \texttt{std::array} values are are equal by testing wether the corresponding elements
are equal. The first $2$ parameters of the functor determine the inputs to the computation (2-dimensional vectors) and
the last parameter determines how the elements are compared.
\begin{verbatim}
// Base of 2,1 functor.
template<typename LEFT_OPERAND, typename RIGHT_OPERAND, typename EQUAL_TO, typename ENABLED = void>
struct equal_to_functor;

// 2,1 functor
template<size_t LEFT_ELEMENT, size_t LEFT_NUMBER_OF_ELEMENTS,
         size_t RIGHT_ELEMENT, size_t RIGHT_NUMBER_OF_ELEMENTS,
         typename EQUAL_TO>
struct equal_to_functor<std::array<LEFT_ELEMENT, LEFT_NUMBER_OF_ELEMENTS>,
                        std::array<RIGHT_ELEMENT, RIGHT_NUMBER_OF_ELEMENTS>,
                        EQUAL_TO,
                        std::enable_if_t<std::is_same_v<LEFT_ELEMENT, RIGHT_ELEMENT> &&
                                         LEFT_NUMBER_OF_ELEMENTS == 2                &&
                                         RIGHT_NUMBER_OF_ELEMENTS == 2>>
{
   using left_operand_type = std::array<LEFT_ELEMENT, LEFT_DIMENSIONALITY>;
   using right_operand_type = std::array<RIGHT_ELEMENT, RIGHT_DIMENSIONALITY>;
   using equal_to_type = EQUAL_TO;
   auto operator(const left_operand_type& u, const right_operand_type& v, equal_to_type equal_to) const
   {
     return equal_to(u[0], v[0])
         && equal_to(u[1], v[1]);
   }
};
\end{verbatim}
%\end{example}

\noindent{}The provided \texttt{operator()} shall be qualified as \texttt{noexcept} and \texttt{constexpr} if possible.\newline

\noindent{}$n,m$-functors are frequently (partial) specializations of a base functor.
The base functor shall provide a \texttt{ENABLED} template parameter with a default value \texttt{void} which used to perform SFINAE.
\noindent{}The return value of the \texttt{operator()} represents                the operation
$f : V_1 \times \ldots \times V_n \times P_1 \times \ldots \times P_m \rightarrow(partial) R$.
where $V_i$ is the type of the $i$-th input and $P_j$ the type of the $j$-th parameter. $R$ is
the result of the computation.

\subsection{\textit{concept $m$-BinaryFunctor}}
An \textit{$m$-BinaryFunctor} is a \textit{$2,m$-Ary-Functor}.

\subsection{\textit{concept $m$-UnaryFunctor}}
An \textit{$m$-UnaryFunctor} is a \textit{$1,m$-Ary-Functor}.

\subsection{\textit{concept $m$-NullaryFunctor}}
An \textit{$m$-NullaryFunctor} is a \textit{$0,m$-Ary-Functor}.
